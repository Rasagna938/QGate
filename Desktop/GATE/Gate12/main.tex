%iffalse
\let\negmedspace\undefined
\let\negthickspace\undefined
\documentclass[journal,12pt,onecolumn]{IEEEtran}
\usepackage{cite}
\usepackage{amsmath,amssymb,amsfonts,amsthm}
\usepackage{algorithmic}
\usepackage{graphicx}
\usepackage{textcomp}
\usepackage{xcolor}
\usepackage{txfonts}
\usepackage{listings}
\usepackage{enumitem}
\usepackage{mathtools}
\usepackage{gensymb}
\usepackage{comment}
\usepackage[breaklinks=true]{hyperref}
\usepackage{tkz-euclide} 
\usepackage{listings}

\usepackage{booktabs}
\usepackage{pgfplots}
\usepackage{gvv}                                        
\usepackage[latin1]{inputenc}     
\usepackage{xparse}
\usepackage{color}                                            
\usepackage{array}                                            
\usepackage{longtable}                                       
\usepackage{calc}                                             
\usepackage{multirow}
\usepackage{multicol}
\usepackage{hhline}                                           
\usepackage{ifthen}                                           
\usepackage{lscape}
\usepackage{tabularx}
\usepackage{array}
\usetikzlibrary{patterns}
\usepackage{siunitx}
\pagestyle{empty}
\usetikzlibrary{calc}
\usepackage[margin=1in]{geometry}
\usepackage{pgffor}
\usepackage{float}
\usepackage{pgf-pie}
\newtheorem{theorem}{Theorem}[section]
\newtheorem{problem}{Problem}
\newtheorem{proposition}{Proposition}[section]
\newtheorem{lemma}{Lemma}[section]
\newtheorem{corollary}[theorem]{Corollary}
\newtheorem{example}{Example}[section]
\newtheorem{definition}[problem]{Definition}
\newcommand{\BEQA}{\begin{eqnarray}}
\newcommand{\EEQA}{\end{eqnarray}}
\newcommand{\define}{\stackrel{\triangle}{=}}
\theoremstyle{remark}
\newtheorem{rem}{Remark}
% Marks the beginning of the document
\pgfplotsset{compat=1.18}
\begin{document}



\bibliographystyle{IEEEtran}
\vspace{3cm}


\title{2024-ST-'53-65'}
\author{EE24BTECH11023}
%\maketitle
%\newpage
%\bigskip
\maketitle




{\let\newpage\relax\maketitle}

\renewcommand{\thefigure}{\theenumi}
\renewcommand{\thetable}{\theenumi}
\setlength{\intextsep}{10pt} % Space between text and floats


\numberwithin{equation}{enumi}
\numberwithin{figure}{enumi}
\renewcommand{\thetable}{\theenumi}
\begin{enumerate}
    \item Let $\{X_n\}_{n \geq 1}$ be a time homogeneous discrete-time Markov chain with state space $\{0, 1, 2\}$ and transition probability matrix
\[
    P = \begin{pmatrix} 
        \frac{1}{2} & \frac{1}{2} & 0 \\ 
        \frac{1}{2} & \frac{1}{2} & 0 \\ 
        \frac{1}{3} & \frac{1}{3} & \frac{1}{3} 
    \end{pmatrix}
\]
    Which of the following statements is/are true?
    \begin{enumerate}
        \item 0 and 1 are recurrent states.
        \item 2 is a transient state.
        \item The Markov chain has a unique stationary distribution.
        \item The Markov chain is irreducible.
    \end{enumerate}

    \item Let $X_1, X_2, \dots, X_n$ be a random sample of size $n (\geq 2)$ from a population having Poisson distribution with mean $\lambda$, where $\lambda > 0$ is an unknown parameter. If $T_1 = \bar{X}$ and $T_2 = \sqrt{\frac{1}{n-1} \sum_{i=1}^{n} (X_i - \bar{X})^2}$, where $\bar{X} = \frac{1}{n} \sum_{i=1}^{n} X_i$, then which of the following statements is/are true?
    \begin{enumerate}
        \item $T_1$ is an unbiased estimator of $\lambda$.
        \item $T_2$ is an unbiased estimator of $\sqrt{\lambda}$.
        \item $T_2^2$ is an unbiased estimator of $\lambda$.
        \item Both $T_1$ and $T_2$ are estimators of $\lambda$ as well as $\lambda^2$.
    \end{enumerate}

    \item Let $X_1, X_2, X_3$ be a random sample of size 3 from a population having Bernoulli distribution with parameter $p$, where $p \in (0,1)$ is unknown. Define
\[
    T_1(X_i, X_j, X_k) = X_i - X_j(1 - X_k), \quad T_2(X_i, X_j, X_k) = \frac{1}{2} (X_i X_j + X_j X_k),
\]
    for $i, j, k = 1, 2, 3$ with $i \neq j \neq k$. Let $x_1, x_2, x_3$ denote realizations from the random sample. Then which of the following statements is/are true?
    \begin{enumerate}
        \item $T_1(X_1, X_2, X_3)$ has the same distribution as $T_1(X_2, X_3, X_1)$, but $T_1(x_1, x_2, x_3) \neq T_1(x_2, x_3, x_1)$ for some realization $x_1, x_2, x_3$.
        \item $T_2(X_1, X_2, X_3)$ and $T_2(X_3, X_2, X_1)$ are both unbiased estimators for $p^2$.
        \item $T_1(X_1, X_2, X_3)$ and $T_1(X_2, X_3, X_1)$ are both unbiased estimators for $p^2$, and $T_1(x_1, x_2, x_3) = T_1(x_2, x_3, x_1)$ for all realizations $x_1, x_2, x_3$.
        \item $T_2(x_1, x_2, x_3) = T_2(x_2, x_3, x_1)$ for all realizations $x_1, x_2, x_3$.
    \end{enumerate}

    \item Let $X_1, X_2, \dots, X_n$ be a random sample of size $n (\geq 2)$ from a population having probability density function
\[
    f(x; \lambda) = \begin{cases} 
        \frac{1}{\lambda} e^{-x/\lambda}, & x > 0 \\ 
        0, & \text{otherwise} 
    \end{cases}
\]
    where $\lambda > 0$ is an unknown parameter. Let $T_1 = \sum_{i=1}^{n} X_i$ and $T_2 = \left(\sum_{i=1}^{n} X_i\right)^{-1}$. For any positive integer $v$ and any $\alpha \in (0, 1)$, let $X_{v,\alpha}^2$ denote the $(1 - \alpha)$-th quantile of the central chi-square distribution with $v$ degrees of freedom. Consider testing $H_0: \lambda = \lambda_0$ against $H_1: \lambda > \lambda_0$. Then which of the following tests is/are uniformly most powerful at level $\alpha$?
    \begin{enumerate}
        \item $H_0$ is rejected if $\frac{2}{\lambda_0}T_1  > X_{2n, \alpha}^2$.
        \item $H_0$ is rejected if $\frac{2}{\lambda_0}T_1> X_{2n, 1 - \alpha}^2$.
        \item $H_0$ is rejected if $\frac{\lambda_0}{2}T_1 > X_{2n, \alpha}^2$.
        \item $H_0$ is rejected if $\frac{\lambda_0}{2}T_1> X_{2n, 1 - \alpha}^2$.
    \end{enumerate}

    \item Let $\{1, 6, 5, 3\}$ and $\{11, 7, 15, 4\}$ be realizations of two independent random samples of size 4 from two separate populations having cumulative distribution functions $F(\cdot)$ and $G(\cdot)$, respectively, and probability density functions $f(\cdot)$ and $g(\cdot)$, respectively. To test $H_0: F(t) = G(t)$ for all $t$ against $H_1: F(t) \geq G(t)$ with strict inequality for some $t$, let $U_{MW}$ denote the Mann-Whitney U-test statistic. Let, under $H_0$, $P(U_{MW} > 12) \leq 0.10$, $P(U_{MW} > 14) \leq 0.05$, $P(U_{MW} > 15) \leq 0.025$, and $P(U_{MW} > 16) \leq 0.01$. Then,based on the given data, which of the following statements is/are true?
    \begin{enumerate}
        \item $H_0$ is rejected at level 0.10.
        \item $H_0$ is rejected at level 0.05.
        \item $H_0$ is rejected at level 0.025.
        \item $H_0$ is rejected at level 0.01.
    \end{enumerate}

    \item Let $(X_1, X_2, X_3)$ have $N_3(\mu, \Sigma)$ distribution with $\mu = \begin{pmatrix} 2 \\ -3 \\ 1 \end{pmatrix}$ and $\Sigma = \begin{pmatrix} 25 & -2 & 4 \\ -2 & 4 & 1 \\ 4 & 1 & 9 \end{pmatrix}$. For which of the following vectors $a$, $X_2$ and $X_2 - a^T \begin{pmatrix} X_1 \\ X_2 \end{pmatrix}$ are independent?
    \begin{enumerate}
        \item $a = \begin{pmatrix}  0 \\ 1 \end{pmatrix}$
        \item $a = \begin{pmatrix}  -1 \\ -1 \end{pmatrix}$
        \item $a = \begin{pmatrix}  0 \\ 2 \end{pmatrix}$
        \item $a = \begin{pmatrix}  2 \\ 2 \end{pmatrix}$
    \end{enumerate}

    \item Let $A$ be a $2 \times 2$ real matrix such that the trace of $A$ is 5 and the determinant of $A$ is 6. If the characteristic polynomial of $(A + I_2)^{-1}$ is $x^2 - bx + c$, where $I_2$ is the $2 \times 2$ identity matrix, then $\frac{b}{c}$ equals {\underline{\hspace{2cm}}} (in integer).

    \item Let $\{X_n\}_{n \geq 1}$ be a time homogeneous discrete-time Markov chain with state space $\{0, 1, 2\}$ and transition probability matrix
\[
    P = \begin{pmatrix}
        0 & \frac{1}{2} & \frac{1}{2} \\
        \frac{1}{2} & \frac{1}{4} & \frac{1}{4} \\
        \frac{1}{2} & \frac{1}{4} & \frac{1}{4}
    \end{pmatrix}
\]
    If $P(X_0 = 0) = P(X_0 = 1) = \frac{1}{4}$, then $32 E(X_2)$ equals  {\underline{\hspace{2cm}}}  (integer).


    \item Let $(X, Y)$ be a random vector having joint moment generating function given by

    $$M_{X,Y}(u, v) = \frac{e^{\frac{u^2}{2}}}{(1 - 2v)^3}, \quad -\infty < u < \infty, -\infty < v < \frac{1}{2}.$$

    Then $E(\frac{6X^2}{Y})$ equals {\underline{\hspace{2cm}}} (rounded off to two decimal places).

    \item Let $\{N(t)\}_{t \geq 0}$ be a Poisson process with rate $\lambda$, where $\lambda > 0$ is an unknown parameter. Starting from the origin, an intercity road has $N(t)$ number of potholes up to a distance of $t$ kilometers. Starting from the origin, potholes are found at the following distances (in kilometers):
\[
    0.9, 1.3, 1.8, 2.7, 3.4, 4.1, 4.7, 5.5, 6.2, 6.8, 7.4, 8.1, 8.9, 9.2, 9.7.
\]
    Based on the above data, the method of moment estimate of $\lambda$ equals  {\underline{\hspace{2cm}}} (rounded off to two decimal places).

    \item Let $X_1, X_2, X_3, X_4$ be a random sample of size 4 from a population having uniform distribution over the interval $(0, \theta)$, where $\theta > 0$ is an unknown parameter. Let $X_{(4)} = \max\{X_1, X_2, X_3, X_4\}$. To test $H_0: \theta = 1$ against $H_1: \theta = 0.1$, consider the critical region that rejects $H_0$ if $X_{(4)} < 0.3$. Let $p$ be the probability of the Type-I error. Then $100p$ equals  {\underline{\hspace{2cm}}} (rounded off to two decimal places).

    \item Let a random sample of size 4 from a normal population with unknown mean $\mu$ and variance 1 yield the sample mean of 0.16. Let $\Phi(\cdot)$ be the cumulative distribution function of the standard normal random variable. It is given that $\Phi(2.28) = 0.989$, $\Phi(1.96) = 0.975$, and $\Phi(1.64) = 0.95$. If the likelihood ratio test of size 0.05 is used to test $H_0: \mu = 0$ against $H_1: \mu \neq 0$, then the power of the test at the sample mean equals  {\underline{\hspace{2cm}}}  (rounded off to three decimal places).

    \item Consider the multiple linear regression model
\[
    y_i = \beta_0 + \beta_1 X_{1i} + \beta_2 X_{2i} + \epsilon_i, \quad i = 1, 2, \dots, 25
\]
    where $\beta_0$, $\beta_1$, and $\beta_2$ are unknown parameters, and $\epsilon_i$'s are uncorrelated random errors with mean 0 and finite variance $\sigma^2 > 0$. Let $F_{\alpha, m, n}$ be such that $P(Y > F_{\alpha, m, n}) = \alpha$, where $Y$ is a random variable with an $F$-distribution with $m$ and $n$ degrees of freedom. Suppose that testing
\[
    H_0: \beta_1 = \beta_2 = 0 \quad \text{against} \quad H_1: \text{At least one of } \beta_1, \beta_2 \text{ is not 0}
\]
    involves computing $F_0 = 11\frac{R^2}{1 - R^2}$ and rejecting $H_0$ if the computed value $F_0$ exceeds $F_{\alpha, 2, 22}$. Given that $F_{0.025, 2, 22} = 4.38$ and $F_{0.05, 2, 22} = 3.44$, the smallest value of $R^2$ that would lead to rejection of $H_0$ for $\alpha = 0.05$ equals  {\underline{\hspace{2cm}}}  (rounded off to two decimal places).

    
\end{enumerate}









  \end{document}